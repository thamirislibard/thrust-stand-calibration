\begin{resumo}
%  O resumo deve ressaltar o objetivo, o método, os resultados e as conclusões 
%  do documento. A ordem e a extensão
%  destes itens dependem do tipo de resumo (informativo ou indicativo) e do
%  tratamento que cada item recebe no documento original. O resumo deve ser
%  precedido da referência do documento, com exceção do resumo inserido no
%  próprio documento. (\ldots) As palavras-chave devem figurar logo abaixo do
%  resumo, antecedidas da expressão Palavras-chave:, separadas entre si por
%  ponto e finalizadas também por ponto. O texto pode conter no mínimo 150 e 
%  no máximo 500 palavras, é aconselhável que sejam utilizadas 200 palavras. 
%  E não se separa o texto do resumo em parágrafos.

Escreva aqui seu resumo em parágrafo único, contemplando: (1) contextualização do tema, (2) objetivo do trabalho, (3) metodologia utilizada, (4) principais resultados encontrados e (5) conclusões mais relevantes. O texto deve ser claro, conciso e objetivo, contendo entre 150 e 500 palavras (preferencialmente 200). Use verbos na voz ativa e na terceira pessoa do singular. Evite símbolos e contrações. Não use citações ou referências bibliográficas. O resumo é a parte mais lida do trabalho, portanto deve ser autocontido e permitir a compreensão geral da pesquisa.

 \vspace{\onelineskip}
    
 \noindent
 \textbf{Palavras-chave}: Primeira palavra-chave. Segunda palavra-chave. Terceira palavra-chave.
\end{resumo}
