\begin{apendicesenv}

% Os apêndices são materiais elaborados pelo autor que complementam sua argumentação,
% mas que não são essenciais para a compreensão do texto principal.
% Exemplos: demonstrações matemáticas extensas, códigos-fonte, especificações
% detalhadas, documentação técnica, etc.

\partapendices

\chapter{Primeiro Apêndice}
\label{ap-primeiro}
% Exemplo de figura em apêndice:
% \begin{figure}[h]
% \centering
% \caption[Título curto para sumário]{Título completo da figura com detalhes.}
% \includegraphics[keepaspectratio=true,scale=0.45]{figuras/nome_figura.eps}
% \label{figura-nome}
% \legend{Fonte: Autor.}
% \end{figure}

\chapter{Segundo Apêndice}
\label{ap-segundo}
% Exemplo: Documentação técnica detalhada
% Exemplo: Códigos-fonte extensos
% Exemplo: Demonstrações matemáticas complexas

% Adicione mais capítulos de apêndices conforme necessário
% Lembre-se: apêndices são documentos elaborados pelo autor
% que complementam o trabalho mas não são essenciais ao texto principal

\end{apendicesenv}
