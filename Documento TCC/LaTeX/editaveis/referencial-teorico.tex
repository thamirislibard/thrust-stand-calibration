\chapter{Referencial Teórico}
\label{cap-referencial-teorico}

% Instruções gerais para o Referencial Teórico:
% 1. Este capítulo deve apresentar a fundamentação teórica do seu trabalho
% 2. Organize os conceitos do mais amplo para o mais específico
% 3. Use citações para apoiar seus argumentos
% 4. Crie seções lógicas e conectadas
% 5. Mantenha o foco nos temas relevantes para seu trabalho

\section{Primeiro Conceito Principal}
\label{rt-conceito1}
% Desenvolva aqui o primeiro conceito importante para seu trabalho
% Use subseções se necessário para organizar melhor o conteúdo

\begin{citacao}
O desenvolvimento de modelos para turbulência tem sido um desafio constante na engenharia \cite{bordalo1989}.
\end{citacao}

\section{Segundo Conceito Principal}
\label{rt-conceito2}
% Desenvolva aqui o segundo conceito importante
% Estabeleça conexões com o conceito anterior quando pertinente

\section{Terceiro Conceito Principal}
\label{rt-conceito3}
% Continue desenvolvendo os conceitos relevantes
% Mantenha o foco na relevância para seu trabalho

\section{Resumo do Capítulo}
\label{rt-resumo}
% Sintetize os principais conceitos discutidos
% Estabeleça as conexões entre os diferentes tópicos abordados